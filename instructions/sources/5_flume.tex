\documentclass{article}
\usepackage[a4paper, total={6in, 8in}]{geometry}
\usepackage[utf8]{inputenc}
\usepackage{polski}
\usepackage[polish]{babel}
\usepackage{listings}
\usepackage[T1]{fontenc}

\pagenumbering{gobble}
\lstset{
	language=bash,
	frame=single,
	basicstyle=\tiny,
	literate=*{-}{-}1,
	columns=fullflexible
	}

\begin{document}
\section*{Flume}

Instalacja flume:
\begin{enumerate}
\item utwórz na hdfsie katalog /user/<login>/events, którego właścicielem będzie user flume
\item zaloguj się w cloudera managerze adminsk02:7180 (admin / admin)
\item Kliknij w usługę 'Flume-<twoj\_numer>', a następnie wejdź w zakładkę 'Configuration'
\item Przekopiuj poniższą konfigurację do pola 'Configuration File' uzupełniając odpowiednie pola
\begin{lstlisting}
tier1.sources  = source1
tier1.channels = channel1
tier1.sinks    = sink1

tier1.sources.source1.type     = netcat
tier1.sources.source1.bind     = 0.0.0.0
tier1.sources.source1.port     = 99<twoj_numer>
tier1.sources.source1.channels = channel1
tier1.sources.source1.interceptors = i1 i2
tier1.sources.source1.interceptors.i1.type = org.apache.flume.interceptor.TimestampInterceptor$Builder
tier1.sources.source1.interceptors.i2.type = pl.isa.hadoop.LoggingInterceptor$Builder
tier1.sources.source1.interceptors.i2.logPrefix = LogPrefix
tier1.channels.channel1.type   = memory


tier1.sinks.sink1.type = hdfs
tier1.sinks.sink1.channel = channel1
tier1.sinks.sink1.hdfs.path = /user/<login>/events/log_time=%y%m%d-%H%M
tier1.sinks.sink1.hdfs.filePrefix = events-
tier1.sinks.sink1.hdfs.fileType = DataStream
tier1.sinks.sink1.hdfs.writeFormat = Text


tier1.channels.channel1.capacity = 100
\end{lstlisting}
\item Dodaj do 'Plugin directories' wartość '/home/<login>/flume.plugins.d'
\item Kliknij 'save changes'
\item Znajdź na cloudera managerze maszynę na której działa flume. Utwórz na niej katalog '/home/<login>/flume.plugins.d/interceptors/lib' i wrzuć do niego zbudowanego jara z interceptorami.
\item Kliknij u góry na 'Actions'
\item Kliknij akcję 'Start'
\item uruchom generator transferów './generator.py <adres\_ip\_maszyny\_z\_flumem> 99<twoj\_numer>'
\item wylistuj katalog na hdfsie /user/<login>/events i zobacz czy pojawiają się logi transferów
\end{enumerate}

\pagebreak

Zadania:
\begin{enumerate}
\item Napisz interceptor, który odfiltruje przlewy na kwotę poniżej 200000, a pozostałe przepuści.
\item Zasymuluj opóźnienia w systemie dostarczania logów i zacznij generować przedwczorajsze logi poleceniem: './generator.py <adres\_ip\_maszyny\_z\_flumem> 99<twoj\_numer> 2'. Eventy zaczną trafiać pod nieprawidłowy log\_time. Napisz interceptor, który odkoduje datę z linijki loga i ustawi odpowiedni timestamp w nagłówku. Do parsowania daty możesz skorzystać z klasy SimpleDateFormat. Sparsowana data ma metodę getTime(). Zwraca ona timestamp, który należy ustawić w nagłówku "timestamp".
\end{enumerate}

\end{document}
