\documentclass{article}
\usepackage[a4paper, total={6in, 8in}]{geometry}
\usepackage[utf8]{inputenc}
\usepackage{polski}
\usepackage[polish]{babel}
\usepackage{listings}
\usepackage[T1]{fontenc}

\pagenumbering{gobble}
\lstset{
	language=bash,
	frame=single,
	basicstyle=\tiny,
	literate=*{-}{-}1,
	columns=fullflexible
	}
	
\begin{document}
\section*{Sqoop}

\begin{enumerate}
\item ściągnij mysql-connector https://cdn.mysql.com//Downloads/Connector-J/mysql-connector-java-5.1.46.tar.gz
\item rozpakuj 'tar -xf mysql-connector-java-5.1.46.tar.gz'
\item W rozpakowanym archiwum znajdź plik 'mysql-connector-java-5.1.46-bin.jar' i wrzuć na vm-cluster-node2 do katalogu /var/lib/sqoop
\item na vm-cluster-node2 uruchom import tabeli account\_owners:
\begin{lstlisting}
sqoop import --connect <connect_string> --username <username> --password <password> \
  --table <table> -m <mappers_count> --target-dir <target_dir>
sqoop import --connect jdbc:mysql://vm-cluster-node1:3306/sqoop --username sqoop --password sqoop_pwd \
  --table account_owners -m 1 --target-dir /user/xyz/account_owners
\end{lstlisting}
\item Utwórz w mysqlu na vm-cluster-node1 tabelę '<username>\_transfers':
\begin{lstlisting}
mysql -u root --password=root
use sqoop
CREATE TABLE xyz_transfers (src varchar(255) DEFAULT NULL, dst varchar(255) DEFAULT NULL, \
  amount varchar(255) DEFAULT NULL, execution_date varchar(255));
\end{lstlisting}
\item wyeksportuj transfery do tabeli w mysqlu:
\begin{lstlisting}
sqoop export --connect <connect_string> --username <username> --password <password> \
  --table <table> -m <mappers_count> --export-dir <export_dir>
sqoop export --connect jdbc:mysql://vm-cluster-node1:3306/sqoop --username sqoop --password sqoop_pwd \
  --table xyz_transfers -m 1 --export-dir /user/xyz/transfers
\end{lstlisting}
\end{enumerate}


\end{document}
