\documentclass{article}
\usepackage[a4paper, total={6in, 8in}]{geometry}
\usepackage[utf8]{inputenc}
\usepackage{polski}
\usepackage[polish]{babel}
\usepackage{listings}
\usepackage[T1]{fontenc}

\pagenumbering{gobble}
\lstset{
	language=sql,
	frame=single,
	basicstyle=\tiny,
	literate=*{-}{-}1,
	columns=fullflexible
	}
	
\begin{document}
\section*{Hive UDF}

\begin{tabular}{c | c }
	UDF & przyjmuje wartości z 1 wiersza i zwraca wartość lub strukturę \\
	UDAF & Funkcja agregująca. Przyjmuje wartości z wielu wierszy i zwraca wartość lub strukturę \\
	UDTF & Funkcja tabelaryczna. Przyjmuje wartości z 1 wiersza i może zwrócić więcej niż 1 'wiersz' \\
\end{tabular}

\bigskip


Wrzucenie i uruchomienie udfa:
\begin{lstlisting}
add jar hdfs:///user/vagrant/hive-udfs-1.0-SNAPSHOT.jar;
create function hello AS 'pl.isa.hadoop.HelloWorldUdf';
select hello(src) from transfers limit 10;
\end{lstlisting}

\subsection*{Zadania}

\begin{enumerate}
\item napisz udfa, który przekonwertuje złotówki na dolary
\item napisz udfa, który przyjmie i połączy po przecinku dowolną ilość prymitywnych parametrów
\item napisz udafa, który z tabeli transfers wygeneruje tabelę zysków i strat
\\*
\begin{tabular}{c | c | c }
	src & dst & amount \\ \hline 
	a & b & 5 \\
	a & c & 10 \\
	b & d & 15 \\
\end{tabular}
->
\begin{tabular}{c | c}
	account & amount \\ \hline 
	a & -5 \\
	a & -10 \\
	b & 5 \\
	b & -15 \\
	c & 10 \\
	d & 15 \\
\end{tabular}
\end{enumerate}


\end{document}
