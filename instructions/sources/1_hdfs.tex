\documentclass[11pt]{article}
\usepackage[utf8]{inputenc}
\usepackage{polski}
\usepackage[polish]{babel}
\usepackage{listings}
\usepackage[T1]{fontenc}

\pagenumbering{gobble}

\lstset{
	language=bash,
	frame=single,
	basicstyle=\tiny,
	literate=*{-}{-}1,
	columns=fullflexible
	}

\begin{document}
\section*{Hdfs}

Przydatne komendy hdfs:
\begin{enumerate}
\item ls - listuje katalog
\begin{lstlisting}
hdfs dfs -ls <path>
hdfs dfs -ls /user/xyz
\end{lstlisting}

\item cat - wypisuje zawartość pliku
\begin{lstlisting}
hdfs dfs -cat <path>
hdfs dfs -cat /user/xyz/file.txt
\end{lstlisting}

\item mkdir - tworzy katalog
\begin{lstlisting}
hdfs dfs -mkdir <path>
hdfs dfs -mkdir /user/xyz/subdir
\end{lstlisting}
 
\item rm - usuwa plik
\begin{lstlisting}
hdfs dfs -rm <path>
hdfs dfs -rm /user/xyz/file.txt
\end{lstlisting}

\item chmod - zmienia prawa dostępu do pliku
\begin{lstlisting}
hdfs dfs -chmod <mode> <path>
hdfs dfs -chmod 640 /user/xyz/file.txt
\end{lstlisting}

\item chown - zmienia właściciela pliku
\begin{lstlisting}
hdfs dfs -chown <owner>:<group> <path>
hdfs dfs -chown xyz:xyz /user/xyz/file.txt
\end{lstlisting}

\item cp - kopiuje plik na hdfsie
\begin{lstlisting}
hdfs dfs -cp <source> <target>
hdfs dfs -cp /user/xyz/file /user/xyz/file.copy
\end{lstlisting}

\pagebreak

\item mv - przenosi plik na hdfsie
\begin{lstlisting}
hdfs dfs -mv <source> <target>
hdfs dfs -mv /user/xyz/file /user/xyz/file.new
\end{lstlisting}

\item put - umieszka lokalny plik na hdfsie
\begin{lstlisting}
hdfs dfs -put <local_path> <hdfs_path>
hdfs dfs -put file.txt /user/xyz
\end{lstlisting}

\item get - pobiera plik z hdfsa
\begin{lstlisting}
hdfs dfs -get <hdfs_path> <local_path>
hdfs dfs -get /user/xyz/file.txt .
\end{lstlisting}

\item touchz - tworzy pusty plik
\begin{lstlisting}
hdfs dfs -touchz <path>
hdfs dfs -touchz /usr/xyz/file.txt
\end{lstlisting}

\item appendToFile - przepisuje plik do końca pliku na hdfsie
\begin{lstlisting}
hdfs dfs -appendToFile <local_path> <hdfs_path>
hdfs dfs -appendToFile abc.txt /user/xyz/file.txt
\end{lstlisting}

\end{enumerate}

Jeżeli klaster nie jest zabezpieczony kerberosem, to możemy wykorzystać zmienną środowiskową HADOOP\_USER\_NAME, żeby ustawić użytkownika z którego wykonane zostanie polecenie, np.
\begin{lstlisting}
HADOOP_USER_NAME=hdfs hdfs dfs -mkdir /katalog
\end{lstlisting}
Odpowiednikiem linuxowego roota na hdfsie jest użytkownik 'hdfs'

\pagebreak

\subsection*{Zadania}

\begin{enumerate}
\item na hostach adminsk0[1-4] załóż sobie usera i katalog domowy
\item Utwórz na hdfsie katalogi z prawami dostępu:\\*

\begin{tabular}{c | c | c | c}
	ścieżka & user & grupa & prawa dostępu \\ \hline 
	/user/<login> & <login> & <login> & 755 \\
	/user/<login>/inputs & <login> & <login> & 755 \\
	/user/<login>/outputs & <login> & <login> & 755 \\
	/user/<login>/alien & <login> & <login> & 755 \\
\end{tabular}

\item wrzuć pliki loremipsum i transaction\_logs do /user/<login>/inputs
\item zmień prawa do odczytu do katalogu /user/<login>/alien tak, żeby tylko 'alien' mógł odczytać jego zawartość
\item spróbuj odczytać ww. katalog jako user ’<login>’
\item spróbuj odczytać ww. katalog jako user ’hdfs’
\end{enumerate}

\end{document}
