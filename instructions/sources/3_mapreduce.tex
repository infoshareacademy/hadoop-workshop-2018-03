\documentclass{article}
\usepackage[a4paper, total={6in, 8in}]{geometry}
\usepackage[utf8]{inputenc}
\usepackage{polski}
\usepackage[polish]{babel}
\usepackage{listings}
\usepackage[T1]{fontenc}

\pagenumbering{gobble}
\lstset{
	language=bash,
	frame=single,
	basicstyle=\tiny,
	columns=fullflexible
	}
	
\begin{document}
\section*{Mapreduce}

Uruchomienie joba mapreduce:
\begin{enumerate}
\item Zbuduj jara i wrzuć na maszynę vm-cluster-node2
\item Zaloguj się na vm-cluster-node2
\item uruchomienie joba
	\begin{lstlisting}[]
hadoop jar <sciezka_do_jara> <pakiet.Klasa> <katalog_wejsciowy> <katalog_wyjsciowy>
hadoop jar mapreduce-jobs-1.0-SNAPSHOT.jar pl.isa.hadoop.WordCount /user/xyz/loremipsum /user/xyz/outputs/output-1
\end{lstlisting}

Uwaga: Żeby polecenie się wykonało <katalog\_wyjsciowy> nie może istnieć!
\end{enumerate}

Zadania:
\begin{enumerate}
\item policz literki w tekscie loremipsum
\item posortuj policzone literki po ilości wystąpień
\item posumuj kwotę z pliku 'transfers' pogrupowaną po rachunku źródłowym
\item posortuj przelewy po kwocie korzystając z wielu reduerów. Napisz Partitioner, który odpowiednio przydzieli przelew do reducera. Możesz założyć, że kwota przelewu to liczba z przedzialu <1, 1000000>
\item połącz posumowane przelewy z danymi klientów z pliku 'clients'. Warto skorzystać z klasy MultipleInputs
\end{enumerate}

\end{document}
