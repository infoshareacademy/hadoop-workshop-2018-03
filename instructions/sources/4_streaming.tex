\documentclass{article}
\usepackage[a4paper, total={6in, 8in}]{geometry}
\usepackage[utf8]{inputenc}
\usepackage{polski}
\usepackage[polish]{babel}
\usepackage{listings}
\usepackage[T1]{fontenc}

\pagenumbering{gobble}
\lstset{
	language=bash,
	frame=single,
	basicstyle=\tiny,
	columns=fullflexible
	}
	
\begin{document}
\section*{Hadoop streaming}

Uruchomienie joba hadoop streaming:
\begin{enumerate}
\item skopiuj skrypty mapper.py i reducer.py na vm-cluster-node2
\item Zaloguj się na vm-cluster-node2
\item uruchomienie joba
\begin{lstlisting}[]
hadoop jar <sciezka_do_jara_hadoop_streaming> -files <lista_plikow_z_mapperem_i_reducerem> \
  -mapper <plik_z_mapperem> -reducer <plik_z_reducerem> -input <katalog_wejsciowy> -output <katalog_wyjsciowy>

hadoop jar /opt/cloudera/parcels/CDH/jars/hadoop-streaming-2.6.0-cdh5.14.0.jar -files mapper.py,reducer.py \
  -mapper mapper.py -reducer reducer.py -input /user/xyz/loremipsum -output /user/xyz/outputs/output-2
\end{lstlisting}

Uwaga: Żeby polecenie się wykonało <katalog\_wyjsciowy> nie może istnieć!
\end{enumerate}

Zadania:
\begin{enumerate}
\item policz literki w tekscie loremipsum
\item posortuj policzone literki po ilości wystąpień
\item Zadanie dodatkowe: napisz mapper i reducer w C\# i użyj w jobie hadoop streaming 
\end{enumerate}

\end{document}
